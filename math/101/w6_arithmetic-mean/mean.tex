%%%%%%%%%%%%%%%%%%%%%%%%%%%%% Define Article %%%%%%%%%%%%%%%%%%%%%%%%%%%%%%%%%%
\documentclass{article}
%%%%%%%%%%%%%%%%%%%%%%%%%%%%%%%%%%%%%%%%%%%%%%%%%%%%%%%%%%%%%%%%%%%%%%%%%%%%%%%

%%%%%%%%%%%%%%%%%%%%%%%%%%%%% Using Packages %%%%%%%%%%%%%%%%%%%%%%%%%%%%%%%%%%
\usepackage{geometry}
\usepackage{graphicx}
\usepackage{amssymb}
\usepackage{amsmath}
\usepackage{amsthm}
\usepackage{empheq}
\usepackage{mdframed}
\usepackage{booktabs}
\usepackage{lipsum}
\usepackage{graphicx}
\usepackage{color}
\usepackage{psfrag}
\usepackage{pgfplots}
\usepackage{bm}
%%%%%%%%%%%%%%%%%%%%%%%%%%%%%%%%%%%%%%%%%%%%%%%%%%%%%%%%%%%%%%%%%%%%%%%%%%%%%%%

% Other Settings

%%%%%%%%%%%%%%%%%%%%%%%%%% Page Setting %%%%%%%%%%%%%%%%%%%%%%%%%%%%%%%%%%%%%%%
\geometry{a4paper}

%%%%%%%%%%%%%%%%%%%%%%%%%% Define some useful colors %%%%%%%%%%%%%%%%%%%%%%%%%%
\definecolor{ocre}{RGB}{243,102,25}
\definecolor{mygray}{RGB}{243,243,244}
\definecolor{deepGreen}{RGB}{26,111,0}
\definecolor{shallowGreen}{RGB}{235,255,255}
\definecolor{deepBlue}{RGB}{61,124,222}
\definecolor{shallowBlue}{RGB}{235,249,255}
%%%%%%%%%%%%%%%%%%%%%%%%%%%%%%%%%%%%%%%%%%%%%%%%%%%%%%%%%%%%%%%%%%%%%%%%%%%%%%%

%%%%%%%%%%%%%%%%%%%%%%%%%% Define an orangebox command %%%%%%%%%%%%%%%%%%%%%%%%
\newcommand\orangebox[1]{\fcolorbox{ocre}{mygray}{\hspace{1em}#1\hspace{1em}}}
%%%%%%%%%%%%%%%%%%%%%%%%%%%%%%%%%%%%%%%%%%%%%%%%%%%%%%%%%%%%%%%%%%%%%%%%%%%%%%%

%%%%%%%%%%%%%%%%%%%%%%%%%%%% English Environments %%%%%%%%%%%%%%%%%%%%%%%%%%%%%
\newtheoremstyle{mytheoremstyle}{3pt}{3pt}{\normalfont}{0cm}{\rmfamily\bfseries}{}{1em}{{\color{black}\thmname{#1}~\thmnumber{#2}}\thmnote{\,--\,#3}}
\newtheoremstyle{myproblemstyle}{3pt}{3pt}{\normalfont}{0cm}{\rmfamily\bfseries}{}{1em}{{\color{black}\thmname{#1}~\thmnumber{#2}}\thmnote{\,--\,#3}}
\theoremstyle{mytheoremstyle}
\newmdtheoremenv[linewidth=1pt,backgroundcolor=shallowGreen,linecolor=deepGreen,leftmargin=0pt,innerleftmargin=20pt,innerrightmargin=20pt,]{theorem}{Theorem}[section]
\theoremstyle{mytheoremstyle}
\newmdtheoremenv[linewidth=1pt,backgroundcolor=shallowBlue,linecolor=deepBlue,leftmargin=0pt,innerleftmargin=20pt,innerrightmargin=20pt,]{definition}{Definition}[section]
\theoremstyle{myproblemstyle}
\newmdtheoremenv[linecolor=black,leftmargin=0pt,innerleftmargin=10pt,innerrightmargin=10pt,]{problem}{Problem}[section]
%%%%%%%%%%%%%%%%%%%%%%%%%%%%%%%%%%%%%%%%%%%%%%%%%%%%%%%%%%%%%%%%%%%%%%%%%%%%%%%

%%%%%%%%%%%%%%%%%%%%%%%%%%%%%%% Plotting Settings %%%%%%%%%%%%%%%%%%%%%%%%%%%%%
\usepgfplotslibrary{colorbrewer}
\pgfplotsset{width=8cm,compat=1.9}
%%%%%%%%%%%%%%%%%%%%%%%%%%%%%%%%%%%%%%%%%%%%%%%%%%%%%%%%%%%%%%%%%%%%%%%%%%%%%%%

%%%%%%%%%%%%%%%%%%%%%%%%%%%%%%% Title & Author %%%%%%%%%%%%%%%%%%%%%%%%%%%%%%%%
\title{Arithmetic Mean}
\author{Halil Yiğit KOÇHAN}
\date{December 12, 2023}
%%%%%%%%%%%%%%%%%%%%%%%%%%%%%%%%%%%%%%%%%%%%%%%%%%%%%%%%%%%%%%%%%%%%%%%%%%%%%%%

\begin{document}
    \maketitle

\section{Aritmetik Ortalama}

$$ x_1 + ... + x_n $$

$$ A.O = \frac{x_1 + ... + x_n}{n} $$

\begin{problem}[4 tane sayının aritmetik ortalaması 20 ise bu sayılara hangi sayı eklenirse aritmetik ortamaları 23 olur?]
\end{problem}

\begin{proof}[\textit{ Sol. }]
  \begin{equation*}
    \begin{aligned}[t]
      20 = \frac{x + y + z + w}{4}\\
      x + y + z + w = x' = 80
    \end{aligned}
    \qquad\qquad
    \begin{aligned}[t]
      23 = \frac{x' + h}{5}\\
      x' + h = 115\\
      h = 35
    \end{aligned}
  \end{equation*}
\end{proof}

\begin{problem}[\text{[1, 11]} aralığındaki 11 doğal sayıdan 1'i atılıyor. Kalan sayıların ortalaması çıkartılan sayıya eşitse çıkartılan sayı kaçtır?]
\end{problem}

\begin{proof}[\textit{ Sol. }]
  \begin{equation*}
    \begin{aligned}[t]
      \frac{\text{Sayı} - x}{10} &= x\\
      \frac{11 \times 12}{2} &= 66 = \text{Sayı}
    \end{aligned}
    \qquad\qquad
    \begin{aligned}[t]
      \frac{66 - x}{10} &= x\\
      66 - x &= 10x\\
      x &= 6
    \end{aligned}
  \end{equation*}
\end{proof}

\begin{problem}[Notların 100 üzerinden değerlendirildiği sistemde, ilk sınavdan 40 alan bir öğrenci en az kaç sınava girerse ortalaması 95 olur?]
\end{problem}

\begin{proof}[\textit{ Sol. }]
  \begin{equation*}
    \begin{aligned}[t]
      \frac{40 + 100x}{x + 1} = 95
    \end{aligned}
    \qquad\qquad
    \begin{aligned}[t]
      &95x + 95 = 40 + 100x\\
      &x = 11
    \end{aligned}
  \end{equation*}
\end{proof}

\section{Geometrik Ortalama}

$$ \sqrt[n]{x_1 \times x_2 \times x_3 \times ... \times x_n} $$

\begin{problem}[a ile b'nin $ G.O = 2 $, b ile c'nin $ G.O = \sqrt{2} $, a ile c'nin $ G.O = 2\sqrt{2} $ ise $ a, b, c $'nin $ G.O = ? $]
\end{problem}

\begin{proof}[\textit{ Sol. }]
  \begin{equation*}
    \begin{aligned}[t]
      \sqrt{a \times b} &= 2\\
      \sqrt{b \times c} &= \sqrt{2}\\
      \sqrt{c \times a} &= 2\sqrt{2}
    \end{aligned}
    \qquad\qquad
    \begin{aligned}[t]
      a \times b &= 4\\
      b \times c &= 2\\
      c \times a &= 8
    \end{aligned}
    \qquad\qquad
    \begin{aligned}[t]
      a^2 \times b^2 \times c^2 &= 64\\
      a \times b \times c &= 8\\
      \sqrt[3]{a \times b \times c} &= 2
    \end{aligned}
  \end{equation*}
\end{proof}

\begin{problem}[Eli, Derya ve Elif; 168 TL'yi sırasıyla $ 3, 5, 6 $ ile orantılı şekilde paylaşıyorlar. Elif = ?]
\end{problem}

\begin{proof}[\textit{ Sol. }]
  \begin{equation*}
    \begin{aligned}[t]
      &\frac{a}{3} = \frac{b}{5} = \frac{c}{6} = k\\
      &a = 3k\\
      &b = 5k\\
      &c = 6k
    \end{aligned}
    \qquad\qquad
    \begin{aligned}[t]
      &14k = 168\\
      &k = 12\\
      &6k = 72
    \end{aligned}
  \end{equation*}
\end{proof}

\begin{problem}[3m uzunluğundaki düzgün bir teli 30cm uzunluğundaki parçalara bölmek 144sn sürüyor. Buna göre aynı teli 25cm'lik parçalara bölmek kaç sn sürer?]
\end{problem}

\begin{proof}[\textit{ Sol. }]
  \begin{equation*}
    \begin{aligned}[t]
      300 \div 30 = 10 \text{ parça, 9 kesim}\\
      300 \div 25 = 12 \text{ parça, 11 kesim}
    \end{aligned}
    \qquad\qquad
    \begin{aligned}[t]
      \begin{cases}
        9 \text{ kesim } &144\\
        11 \text{ kesim } &x
      \end{cases}
      \begin{cases}
        9x = 144 \times 11\\
        x = 16 \times 11 = 176 \text{ sn}
      \end{cases}
    \end{aligned}
  \end{equation*}
\end{proof}

\begin{problem}[310 TL'yi 3 kişi 2, 3 ile doğru 6 ile ters orantılı şekilde paylaşıyorlar. En az alan kaç TL alır?]
\end{problem}

\begin{proof}[\textit{ Sol. }]
  \begin{equation*}
    \begin{aligned}[t]
      \frac{a}{2} = \frac{b}{3} = 6c = k
    \end{aligned}
    \qquad\qquad
    \begin{aligned}[t]
      a &= 2k\\
      b &= 3k\\
      c &= \frac{6}{k}
    \end{aligned}
    \qquad\qquad
    \begin{aligned}[t]
      \frac{2k + 3k + k}{6} &= 31k\\
      \frac{31k}{6} &= 310\\
      k &= 60
    \end{aligned}
  \end{equation*}
\end{proof}

\begin{problem}[6 işçinin 12 günde yaptığını 9 işçi kaç günde yapar?]
\end{problem}

\begin{proof}[\textit{ Sol. }]
  \begin{equation*}
    \begin{aligned}[t]
      &6 => 12\\
      &9 => x
    \end{aligned}
    \qquad\qquad
    \begin{aligned}[t]
      72 &= 9x\\
      x &= 8
    \end{aligned}
  \end{equation*}
\end{proof}

\begin{problem}[30 kişilik gruba 48 gün yetecek yiyecek vardır. 8 gün sonra bu gruptan 6 kişi ayrılırsa kalan yiyecek kalan gruba kaç gün yeter?]
\end{problem}

\begin{proof}[\textit{ Sol. }]
  \begin{equation*}
    \begin{aligned}[t]
      48 - 8 = 40 \text{ günlük}\\
      30 => 40\\
      24 => x
    \end{aligned}
    \qquad\qquad
    \begin{aligned}[t]
      30 \times 40 = 24x\\
      x = 50
    \end{aligned}
  \end{equation*}
\end{proof}

\end{document}
