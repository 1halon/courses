%%%%%%%%%%%%%%%%%%%%%%%%%%%%% Define Article %%%%%%%%%%%%%%%%%%%%%%%%%%%%%%%%%%
\documentclass{article}
%%%%%%%%%%%%%%%%%%%%%%%%%%%%%%%%%%%%%%%%%%%%%%%%%%%%%%%%%%%%%%%%%%%%%%%%%%%%%%%

%%%%%%%%%%%%%%%%%%%%%%%%%%%%% Using Packages %%%%%%%%%%%%%%%%%%%%%%%%%%%%%%%%%%
\usepackage{geometry}
\usepackage{graphicx}
\usepackage{amssymb}
\usepackage{amsmath}
\usepackage{amsthm}
\usepackage{empheq}
\usepackage{mdframed}
\usepackage{booktabs}
\usepackage{lipsum}
\usepackage{graphicx}
\usepackage{color}
\usepackage{psfrag}
\usepackage{pgfplots}
\usepackage{bm}
%%%%%%%%%%%%%%%%%%%%%%%%%%%%%%%%%%%%%%%%%%%%%%%%%%%%%%%%%%%%%%%%%%%%%%%%%%%%%%%

% Other Settings

%%%%%%%%%%%%%%%%%%%%%%%%%% Page Setting %%%%%%%%%%%%%%%%%%%%%%%%%%%%%%%%%%%%%%%
\geometry{a4paper}

%%%%%%%%%%%%%%%%%%%%%%%%%% Define some useful colors %%%%%%%%%%%%%%%%%%%%%%%%%%
\definecolor{ocre}{RGB}{243,102,25}
\definecolor{mygray}{RGB}{243,243,244}
\definecolor{deepGreen}{RGB}{26,111,0}
\definecolor{shallowGreen}{RGB}{235,255,255}
\definecolor{deepBlue}{RGB}{61,124,222}
\definecolor{shallowBlue}{RGB}{235,249,255}
%%%%%%%%%%%%%%%%%%%%%%%%%%%%%%%%%%%%%%%%%%%%%%%%%%%%%%%%%%%%%%%%%%%%%%%%%%%%%%%

%%%%%%%%%%%%%%%%%%%%%%%%%% Define an orangebox command %%%%%%%%%%%%%%%%%%%%%%%%
\newcommand\orangebox[1]{\fcolorbox{ocre}{mygray}{\hspace{1em}#1\hspace{1em}}}
%%%%%%%%%%%%%%%%%%%%%%%%%%%%%%%%%%%%%%%%%%%%%%%%%%%%%%%%%%%%%%%%%%%%%%%%%%%%%%%

%%%%%%%%%%%%%%%%%%%%%%%%%%%% English Environments %%%%%%%%%%%%%%%%%%%%%%%%%%%%%
\newtheoremstyle{mytheoremstyle}{3pt}{3pt}{\normalfont}{0cm}{\rmfamily\bfseries}{}{1em}{{\color{black}\thmname{#1}~\thmnumber{#2}}\thmnote{\,--\,#3}}
\newtheoremstyle{myproblemstyle}{3pt}{3pt}{\normalfont}{0cm}{\rmfamily\bfseries}{}{1em}{{\color{black}\thmname{#1}~\thmnumber{#2}}\thmnote{\,--\,#3}}
\theoremstyle{mytheoremstyle}
\newmdtheoremenv[linewidth=1pt,backgroundcolor=shallowGreen,linecolor=deepGreen,leftmargin=0pt,innerleftmargin=20pt,innerrightmargin=20pt,]{theorem}{Theorem}[section]
\theoremstyle{mytheoremstyle}
\newmdtheoremenv[linewidth=1pt,backgroundcolor=shallowBlue,linecolor=deepBlue,leftmargin=0pt,innerleftmargin=20pt,innerrightmargin=20pt,]{definition}{Definition}[section]
\theoremstyle{myproblemstyle}
\newmdtheoremenv[linecolor=black,leftmargin=0pt,innerleftmargin=10pt,innerrightmargin=10pt,]{problem}{Problem}[section]
%%%%%%%%%%%%%%%%%%%%%%%%%%%%%%%%%%%%%%%%%%%%%%%%%%%%%%%%%%%%%%%%%%%%%%%%%%%%%%%

%%%%%%%%%%%%%%%%%%%%%%%%%%%%%%% Plotting Settings %%%%%%%%%%%%%%%%%%%%%%%%%%%%%
\usepgfplotslibrary{colorbrewer}
\pgfplotsset{width=8cm,compat=1.9}
%%%%%%%%%%%%%%%%%%%%%%%%%%%%%%%%%%%%%%%%%%%%%%%%%%%%%%%%%%%%%%%%%%%%%%%%%%%%%%%

%%%%%%%%%%%%%%%%%%%%%%%%%%%%%%% Title & Author %%%%%%%%%%%%%%%%%%%%%%%%%%%%%%%%
\title{Radical Numbers}
\author{Halil Yiğit KOÇHAN}
\date{December 19, 2023}
%%%%%%%%%%%%%%%%%%%%%%%%%%%%%%%%%%%%%%%%%%%%%%%%%%%%%%%%%%%%%%%%%%%%%%%%%%%%%%%

\begin{document}
    \maketitle

\section{Köklü Sayılar}

\begin{gather*}
  \sqrt[n]{a^m} = \frac{a^m}{n}\\
  \sqrt{a^2} = a\\
  \sqrt[3]{a^3} = a\\
  \sqrt[m]{\sqrt[n]{\sqrt[x]{\sqrt[a]{a}}}} = \sqrt[mnx]{a}\\
  \sqrt[n]{x} \times \sqrt[n]{y} = \sqrt[n]{x \times y}\\
  \sqrt{a + b + 2\sqrt{a \times b}} = \sqrt{a} + \sqrt{b}\\
  \sqrt{a + b - 2\sqrt{a \times b}} = \sqrt{a} - \sqrt{b}\\
  \frac{\sqrt[n]{x}}{\sqrt[n]{y}} = \sqrt[n]{\frac{x}{y}}
\end{gather*}

\begin{problem}[$ \sqrt{5 + 2\sqrt{6}} = ? $]
\end{problem}

\begin{proof}[\textit{ Sol. }]
  $$ \sqrt{3} + \sqrt{2} $$
\end{proof}

\subsection{Eşlenik}

\begin{gather*}
  \sqrt{x} -> \sqrt{x}\\
  \sqrt{x + y} = \sqrt{x + y}
  \sqrt{x} + \sqrt{y} = \sqrt{x} - \sqrt{y}
\end{gather*}

\begin{problem}[$ \sqrt{10 - \sqrt{31 + \sqrt{21} + \sqrt{19} - \sqrt{9}}} = ? $]
\end{problem}

\begin{proof}[\textit{ Sol. }]
  $$ \sqrt{10 - 6} = \sqrt{4} = 2 $$
\end{proof}

\begin{problem}[$ \sqrt{18} + \sqrt{50} + \sqrt{72} + \sqrt{8} = ? $]
\end{problem}

\begin{proof}[\textit{ Sol. }]
  $$ 3\sqrt{2} + 5\sqrt{2} + 6\sqrt{2} - 2\sqrt{2} = 12\sqrt{2} = \sqrt{288} $$
\end{proof}

\begin{problem}[$ \sqrt{2 + \sqrt{a}} = 3 $, $ \sqrt{30 + \sqrt{27 + \sqrt{b}}} = 6 $ ise $ b - a = ? $]
\end{problem}

\begin{proof}[\textit{ Sol. }]
  \begin{equation*}
    \begin{aligned}[t]
      2 + \sqrt{a} = 9\\
      \sqrt{a}^2 = 7^2\\
      a = 49
    \end{aligned}
    \qquad\qquad
    \begin{aligned}[t]
      30 + \sqrt{27 + \sqrt{b}} = 36\\
      (27 + \sqrt{b})^2 = 6^2\\
      \sqrt{b^2} = 9^2\\
      b = 81
    \end{aligned}
  \end{equation*}
  $$ b - a = 81 - 49 = 32 $$
\end{proof}

\begin{problem}[$ \frac{\sqrt{0,49} - \sqrt{64} + \sqrt{1,69}}{\sqrt{5} - \sqrt{2}} = ? $]
\end{problem}

\begin{proof}[\textit{ Sol. }]
  \begin{equation*}
    \begin{aligned}[t]
      \frac{0,7 - 8 + 13}{\sqrt{5} - \sqrt{2}} = &\frac{-6}{\sqrt{5} - \sqrt{2}}\\
      &(\sqrt{5} - \sqrt{2})
    \end{aligned}
    \qquad\qquad
    \begin{aligned}[t]
      \frac{-6\sqrt{5} - 6\sqrt{2}}{3} = -2\sqrt{5} - 2\sqrt{3}
    \end{aligned}
  \end{equation*}
\end{proof}

\begin{problem}[$ \frac{3 + \sqrt{3}}{3 - \sqrt{3}} + \frac{3 - \sqrt{3}}{3 + \sqrt{3}} = ? $]
\end{problem}

\begin{proof}[\textit{ Sol. }]
  $$ \frac{9 + 3\sqrt{3} + 3\sqrt{3} + 9 - 3\sqrt{3} - 3\sqrt{3} + 3}{6} = \frac{24}{6} = 4 $$
\end{proof}

\begin{problem}[$ \sqrt{42 + \sqrt{42 + \sqrt{42} + ...}} = ? $ (x) $ (x = \sqrt{42 + \sqrt{42 + ...}})$]
\end{problem}

\begin{proof}[\textit{ Sol. }]
  \begin{equation*}
    \begin{aligned}[t]
      \sqrt{42 + x^2} = x^2\\
      42 + x = x^2\\
      42 = x^2 - x
    \end{aligned}
    \qquad\qquad
    \begin{aligned}[t]
      42 = &x(x - 1)\\
      &7\;\;6
    \end{aligned}
  \end{equation*}
  $$ x = 7 $$
\end{proof}

\begin{problem}[$ \frac{3}{\sqrt{3} - 1} - \frac{1}{\sqrt{4 + 2\sqrt{3}}} = ? $]
\end{problem}

\begin{proof}[\textit{ Sol. }]
  \begin{equation*}
    \begin{aligned}[t]
      \frac{3}{\sqrt{3} - 1} - \frac{1}{\sqrt{3} + 1}\\
      \frac{3\sqrt{3} + 3 - \sqrt{3} + 1}{2}
    \end{aligned}
    \qquad\qquad
    \begin{aligned}[t]
      \frac{2\sqrt{3} + 4}{2} = \sqrt{3} + 2
    \end{aligned}
  \end{equation*}
\end{proof}

\begin{problem}[$ x, y \in {R+} $, $ \frac{x\sqrt{y} + y\sqrt{x}}{\sqrt{x} + \sqrt{y}} = 5 $; $ x \times y = ? $]
\end{problem}

\begin{proof}[\textit{ Sol. }]
  \begin{equation*}
    \begin{aligned}[t]
      \frac{x\sqrt{x \times y} + y \times x - x \times y - y \sqrt{x \times y}}{x - y} = 5\\
      5x - 5y = x \sqrt{x \times y - y \sqrt{x \times y}}
    \end{aligned}
    \qquad\qquad
    \begin{aligned}[t]
      5(x - y) = \sqrt{x \times y(x - y)}\\
      \sqrt{x \times y}^2 = 5^2\\
      x \times y = 25
    \end{aligned}
  \end{equation*}
\end{proof}

\begin{problem}[$ x, y \in {R} $, $ y = \sqrt{x - 5} + \sqrt{2 - x} $ ise $ x $ hangi aralıkta değer alır?]
\end{problem}

\begin{proof}[\textit{ Sol. }]
  \begin{equation*}
    \begin{aligned}[t]
      x -5 \ge 0,\quad x >= 5\\
      2 -x \ge 0,\quad 2 >= x\\
      (-\infty, 2] \cup [5, \infty)\\
      \sqrt{5}, \sqrt{7}, \sqrt{3}, \sqrt{3} < \sqrt{5} < \sqrt{7}\\
    \end{aligned}
    \qquad\qquad
    \begin{aligned}[t]
      a = \sqrt[3]{2} = \sqrt[12]{16}\\
      b = \sqrt[4]{3} = \sqrt[12]{27}\\
      c = \sqrt[6]{7} = \sqrt[12]{49}\\
      c > b > a
    \end{aligned}
  \end{equation*}
\end{proof}

\end{document}
