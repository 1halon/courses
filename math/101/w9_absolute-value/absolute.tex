%%%%%%%%%%%%%%%%%%%%%%%%%%%%% Define Article %%%%%%%%%%%%%%%%%%%%%%%%%%%%%%%%%%
\documentclass{article}
%%%%%%%%%%%%%%%%%%%%%%%%%%%%%%%%%%%%%%%%%%%%%%%%%%%%%%%%%%%%%%%%%%%%%%%%%%%%%%%

%%%%%%%%%%%%%%%%%%%%%%%%%%%%% Using Packages %%%%%%%%%%%%%%%%%%%%%%%%%%%%%%%%%%
\usepackage{geometry}
\usepackage{graphicx}
\usepackage{amssymb}
\usepackage{amsmath}
\usepackage{amsthm}
\usepackage{empheq}
\usepackage{mdframed}
\usepackage{booktabs}
\usepackage{lipsum}
\usepackage{graphicx}
\usepackage{color}
\usepackage{psfrag}
\usepackage{pgfplots}
\usepackage{bm}
%%%%%%%%%%%%%%%%%%%%%%%%%%%%%%%%%%%%%%%%%%%%%%%%%%%%%%%%%%%%%%%%%%%%%%%%%%%%%%%

% Other Settings

%%%%%%%%%%%%%%%%%%%%%%%%%% Page Setting %%%%%%%%%%%%%%%%%%%%%%%%%%%%%%%%%%%%%%%
\geometry{a4paper}

%%%%%%%%%%%%%%%%%%%%%%%%%% Define some useful colors %%%%%%%%%%%%%%%%%%%%%%%%%%
\definecolor{ocre}{RGB}{243,102,25}
\definecolor{mygray}{RGB}{243,243,244}
\definecolor{deepGreen}{RGB}{26,111,0}
\definecolor{shallowGreen}{RGB}{235,255,255}
\definecolor{deepBlue}{RGB}{61,124,222}
\definecolor{shallowBlue}{RGB}{235,249,255}
%%%%%%%%%%%%%%%%%%%%%%%%%%%%%%%%%%%%%%%%%%%%%%%%%%%%%%%%%%%%%%%%%%%%%%%%%%%%%%%

%%%%%%%%%%%%%%%%%%%%%%%%%% Define an orangebox command %%%%%%%%%%%%%%%%%%%%%%%%
\newcommand\orangebox[1]{\fcolorbox{ocre}{mygray}{\hspace{1em}#1\hspace{1em}}}
%%%%%%%%%%%%%%%%%%%%%%%%%%%%%%%%%%%%%%%%%%%%%%%%%%%%%%%%%%%%%%%%%%%%%%%%%%%%%%%

%%%%%%%%%%%%%%%%%%%%%%%%%%%% English Environments %%%%%%%%%%%%%%%%%%%%%%%%%%%%%
\newtheoremstyle{mytheoremstyle}{3pt}{3pt}{\normalfont}{0cm}{\rmfamily\bfseries}{}{1em}{{\color{black}\thmname{#1}~\thmnumber{#2}}\thmnote{\,--\,#3}}
\newtheoremstyle{myproblemstyle}{3pt}{3pt}{\normalfont}{0cm}{\rmfamily\bfseries}{}{1em}{{\color{black}\thmname{#1}~\thmnumber{#2}}\thmnote{\,--\,#3}}
\theoremstyle{mytheoremstyle}
\newmdtheoremenv[linewidth=1pt,backgroundcolor=shallowGreen,linecolor=deepGreen,leftmargin=0pt,innerleftmargin=20pt,innerrightmargin=20pt,]{theorem}{Theorem}[section]
\theoremstyle{mytheoremstyle}
\newmdtheoremenv[linewidth=1pt,backgroundcolor=shallowBlue,linecolor=deepBlue,leftmargin=0pt,innerleftmargin=20pt,innerrightmargin=20pt,]{definition}{Definition}[section]
\theoremstyle{myproblemstyle}
\newmdtheoremenv[linecolor=black,leftmargin=0pt,innerleftmargin=10pt,innerrightmargin=10pt,]{problem}{Problem}[section]
%%%%%%%%%%%%%%%%%%%%%%%%%%%%%%%%%%%%%%%%%%%%%%%%%%%%%%%%%%%%%%%%%%%%%%%%%%%%%%%

%%%%%%%%%%%%%%%%%%%%%%%%%%%%%%% Plotting Settings %%%%%%%%%%%%%%%%%%%%%%%%%%%%%
\usepgfplotslibrary{colorbrewer}
\pgfplotsset{width=8cm,compat=1.9}
%%%%%%%%%%%%%%%%%%%%%%%%%%%%%%%%%%%%%%%%%%%%%%%%%%%%%%%%%%%%%%%%%%%%%%%%%%%%%%%

%%%%%%%%%%%%%%%%%%%%%%%%%%%%%%% Title & Author %%%%%%%%%%%%%%%%%%%%%%%%%%%%%%%%
\title{Absolute Value}
\author{Halil Yiğit KOÇHAN}
\date{December 19, 2023}
%%%%%%%%%%%%%%%%%%%%%%%%%%%%%%%%%%%%%%%%%%%%%%%%%%%%%%%%%%%%%%%%%%%%%%%%

\begin{document}
    \maketitle

\begin{gather*}
  |x| =
  \begin{cases}
    x < 0, &-x\\
    x > 0, &x
  \end{cases}\\
  \left|\frac{-1}{2}\right| = \frac{1}{2}\\
  |-x| = |x|\\
  |x - y| = |y - x|\\
  |x \times y| = |x| \times |y|\\
  |\frac{x}{y}| = \frac{|x|}{|y|}\\
  a > 0, |f(x)| = a \text{ ise } f(x) = a, f(x) = -a\\
  a > 0, |f(x)| \le a \text{ ise } -a \le f(x) \le a\\
  a > 0, |f(x)| \ge 0 \text { ise } f(x) \ge a, f(x) \le -a\\
  |x + y| \le |x| + |y|
\end{gather*}

\begin{problem}[$ x > 0 $ ve $ y < 0 $ ise $ |x - y| + |x| + |y| = ? $]
\end{problem}

\begin{proof}[\textit{ Sol. }]
  $$ x - y + x - y = 2x -2y = 2(x - y) $$
\end{proof}

\begin{problem}[$ x < 0 < y $ ise $ \frac{x^2 + 2|xy| + y^2}{|y - x|} = ? $]
\end{problem}

\begin{proof}[\textit{ Sol. }]
  $$ \frac{x^2 + y^2 - 2xy}{y - x} = \frac{(x - y)^2}{y - x} = y - x $$
\end{proof}

\pagebreak
\begin{problem}[$ a, b \in {R} $, $ a - b = 2 $, $ a - |b - a| = 4 $ ise $ a + b = ? $]
\end{problem}

\begin{proof}[\textit{ Sol. }]
  \begin{equation*}
    \begin{aligned}[t]
      |a - b| = 2\\
      |b - a| = |a - b| = 2
    \end{aligned}
    \qquad\qquad
    \begin{aligned}[t]
      a - 2 = 4\\
      a = 6\\
      6 + 4 = 10
    \end{aligned}
  \end{equation*}
\end{proof}

\begin{problem}[$ x < 0 < y $ olmak üzere $ \sqrt{9x^2} - {\sqrt[3]{y^3}} + \sqrt{(x - y)^2} $]
\end{problem}

\begin{proof}[\textit{ Sol. }]
  $$ |3x| - y + |x - y| = -3x - y - x + y = -4x $$
\end{proof}

\begin{problem}[$ |x - 3| = 1905! $ ise $ x $ değerleri toplamı?]
\end{problem}

\begin{proof}[\textit{ Sol. }]
  \begin{equation*}
    \begin{aligned}[t]
      x - 3 = 1905!,\quad x = 1905! + 3
    \end{aligned}
    \qquad\qquad
    \begin{aligned}[t]
      x - 3 = -1905!,\quad x = 1905 + 3
    \end{aligned}
  \end{equation*}
  $$ x = 1905 + 3 -1905! + 3 = 6 $$
\end{proof}

\begin{problem}[$ |x^2 - 16| = |x - 4| $ denklemeni sağlayan $ x $ değerleri toplamı?]
\end{problem}

\begin{proof}[\textit{ Sol. }]
  $$ |x - 4| \times |x + 4| = |x - 4| $$
  \begin{equation*}
    \begin{aligned}[t]
      |x + 4| = 1\\
      x + 4 = 1\\
      x = -3
    \end{aligned}
    \qquad\qquad
    \begin{aligned}[t]
      |x + 4| = -1\\
      x + 4 = -1\\
      x = -5
    \end{aligned}
  \end{equation*}
\end{proof}

\begin{problem}[$ |x - 1| \le 2 $, $ x + y - 3 = 0 $, $ y $'nin en büyük değeri?]
\end{problem}

\begin{proof}[\textit{ Sol. }]
  \begin{equation*}
    \begin{aligned}[t]
      -2 \le x - 1 \le 2\\
      -1 \le x \le 3
    \end{aligned}
    \qquad\qquad
    \begin{aligned}[t]
      x + y = 3\\
      x = -1\\
      y = 4
    \end{aligned}
  \end{equation*}
\end{proof}

\begin{problem}[$ |\frac{2}{x - 3} \le \frac{1}{2} $ eşitliğine sağlayan $ x $ değerleri toplamı?]
\end{problem}

\begin{proof}[\textit{ Sol. }]
  $$ |x - \frac{3}{2} \le 2 $$
  \begin{equation*}
    \begin{aligned}[t]
      -2 \le x \frac{3}{2} \le 2\\
      -4 \le x - 3 \le 4
    \end{aligned}
    \qquad\qquad
    \begin{aligned}[t]
      -1 \le x \le 7\\
      [-1, 7] - \{3\} = 24
    \end{aligned}
  \end{equation*}
\end{proof}

\begin{problem}[$ \frac{|x + 1| - 4}{|x| + 2} \le 0 $ eşitliğini sağlayan $ x $ tamsayı değerlerinin toplamı?]
\end{problem}

\begin{proof}[\textit{ Sol. }]
  \begin{equation*}
    \begin{aligned}[t]
      |x + 1| - 4 \le 0\\
      |x + 1| \le 4
    \end{aligned}
    \qquad\qquad
    \begin{aligned}[t]
      -4 \le x + 1 \le 4\\
      -5 \le x \le 3
    \end{aligned}
  \end{equation*}
  $$ \sum_{}^{}x = 9 $$
\end{proof}

\begin{problem}[$ \sqrt{x^2 + 6x + 9} > 2 $ eşitliğini $ x $ tamsayı değerleri toplamı? (sağlamayan = $ \le 2 $)]
\end{problem}

\begin{proof}[\textit{ Sol. }]
  \begin{equation*}
    \begin{aligned}[t]
      \sqrt{(x - 3)^2}\\
      |x - 3| \le 2
    \end{aligned}
    \qquad\qquad
    \begin{aligned}[t]
      -2 \le x - 3 \le 2\\
      1 \le x \le 5
    \end{aligned}
  \end{equation*}
  $$ \sum_{}^{}x = 15 $$
\end{proof}

\begin{problem}[$ |x + 4| < |x + 6| $, $ Ç.K = ? $]
\end{problem}

\begin{proof}[\textit{ Sol. }]
  \begin{equation*}
    \begin{aligned}[t]
      (x + 4)^2 < (x + 6)^2\\
      x^2 + 8x + 16 < x^2 + 12x + 36
    \end{aligned}
    \qquad\qquad
    \begin{aligned}[t]
      -4x < 20\\
      x > -5
    \end{aligned}
  \end{equation*}
  $$ Ç.K = (-5, \infty) $$
\end{proof}

\begin{problem}[$ 7 \le |2x - 1| \le 13 $ eşitsizliğini sağlayan kaç $ x $ değeri vardır?]
\end{problem}

\begin{proof}[\textit{ Sol. }]
  \begin{equation*}
    \begin{aligned}[t]
      7 \le 2x - 1 \le 13\\
      8 \le 2x \le 14\\
      4 \le x \le 7
    \end{aligned}
    \qquad\qquad
    \begin{aligned}[t]
      7 \le -2 + 1 \le 13\\
      6 \le -2 \le 12\\
      -12 \le 2x \le -6\\
      -6 \le x \le -3
    \end{aligned}
  \end{equation*}
  $$ \sum_{}^{}x = 8 $$
\end{proof}

\end{document}
