%%%%%%%%%%%%%%%%%%%%%%%%%%%%% Define Article %%%%%%%%%%%%%%%%%%%%%%%%%%%%%%%%%%
\documentclass{article}
%%%%%%%%%%%%%%%%%%%%%%%%%%%%%%%%%%%%%%%%%%%%%%%%%%%%%%%%%%%%%%%%%%%%%%%%%%%%%%%

%%%%%%%%%%%%%%%%%%%%%%%%%%%%% Using Packages %%%%%%%%%%%%%%%%%%%%%%%%%%%%%%%%%%
\usepackage{geometry}
\usepackage{graphicx}
\usepackage{amssymb}
\usepackage{amsmath}
\usepackage{amsthm}
\usepackage{empheq}
\usepackage{mdframed}
\usepackage{booktabs}
\usepackage{lipsum}
\usepackage{graphicx}
\usepackage{color}
\usepackage{psfrag}
\usepackage{pgfplots}
\usepackage{bm}
%%%%%%%%%%%%%%%%%%%%%%%%%%%%%%%%%%%%%%%%%%%%%%%%%%%%%%%%%%%%%%%%%%%%%%%%%%%%%%%

% Other Settings

%%%%%%%%%%%%%%%%%%%%%%%%%% Page Setting %%%%%%%%%%%%%%%%%%%%%%%%%%%%%%%%%%%%%%%
\geometry{a4paper}

%%%%%%%%%%%%%%%%%%%%%%%%%% Define some useful colors %%%%%%%%%%%%%%%%%%%%%%%%%%
\definecolor{ocre}{RGB}{243,102,25}
\definecolor{mygray}{RGB}{243,243,244}
\definecolor{deepGreen}{RGB}{26,111,0}
\definecolor{shallowGreen}{RGB}{235,255,255}
\definecolor{deepBlue}{RGB}{61,124,222}
\definecolor{shallowBlue}{RGB}{235,249,255}
%%%%%%%%%%%%%%%%%%%%%%%%%%%%%%%%%%%%%%%%%%%%%%%%%%%%%%%%%%%%%%%%%%%%%%%%%%%%%%%

%%%%%%%%%%%%%%%%%%%%%%%%%% Define an orangebox command %%%%%%%%%%%%%%%%%%%%%%%%
\newcommand\orangebox[1]{\fcolorbox{ocre}{mygray}{\hspace{1em}#1\hspace{1em}}}
%%%%%%%%%%%%%%%%%%%%%%%%%%%%%%%%%%%%%%%%%%%%%%%%%%%%%%%%%%%%%%%%%%%%%%%%%%%%%%%

%%%%%%%%%%%%%%%%%%%%%%%%%%%% English Environments %%%%%%%%%%%%%%%%%%%%%%%%%%%%%
\newtheoremstyle{mytheoremstyle}{3pt}{3pt}{\normalfont}{0cm}{\rmfamily\bfseries}{}{1em}{{\color{black}\thmname{#1}~\thmnumber{#2}}\thmnote{\,--\,#3}}
\newtheoremstyle{myproblemstyle}{3pt}{3pt}{\normalfont}{0cm}{\rmfamily\bfseries}{}{1em}{{\color{black}\thmname{#1}~\thmnumber{#2}}\thmnote{\,--\,#3}}
\theoremstyle{mytheoremstyle}
\newmdtheoremenv[linewidth=1pt,backgroundcolor=shallowGreen,linecolor=deepGreen,leftmargin=0pt,innerleftmargin=20pt,innerrightmargin=20pt,]{theorem}{Theorem}[section]
\theoremstyle{mytheoremstyle}
\newmdtheoremenv[linewidth=1pt,backgroundcolor=shallowBlue,linecolor=deepBlue,leftmargin=0pt,innerleftmargin=20pt,innerrightmargin=20pt,]{definition}{Definition}[section]
\theoremstyle{myproblemstyle}
\newmdtheoremenv[linecolor=black,leftmargin=0pt,innerleftmargin=10pt,innerrightmargin=10pt,]{problem}{Problem}[section]
%%%%%%%%%%%%%%%%%%%%%%%%%%%%%%%%%%%%%%%%%%%%%%%%%%%%%%%%%%%%%%%%%%%%%%%%%%%%%%%

%%%%%%%%%%%%%%%%%%%%%%%%%%%%%%% Plotting Settings %%%%%%%%%%%%%%%%%%%%%%%%%%%%%
\usepgfplotslibrary{colorbrewer}
\pgfplotsset{width=8cm,compat=1.9}
%%%%%%%%%%%%%%%%%%%%%%%%%%%%%%%%%%%%%%%%%%%%%%%%%%%%%%%%%%%%%%%%%%%%%%%%%%%%%%%

%%%%%%%%%%%%%%%%%%%%%%%%%%%%%%% Title & Author %%%%%%%%%%%%%%%%%%%%%%%%%%%%%%%%
\title{Faktöriyel}
\author{Halil Yiğit KOÇHAN}
\date{November 22, 2023}
%%%%%%%%%%%%%%%%%%%%%%%%%%%%%%%%%%%%%%%%%%%%%%%%%%%%%%%%%%%%%%%%%%%%%%%%%%%%%%%

\begin{document}
    \maketitle

\begin{itemize}
  \item 1'den n'ye kadar olan sayılar "$ n! $"
  \item Negatif sayıların faktöriyeli yoktur.
  \item $ 0! = 1 $
  \item $ 1! = 1 $
  \item $ 2! = 2 $
  \item $ 3! = 6 $
  \item $ 4! = 24 $
  \item $ 5! = 120 $
  \item $ 6! = 720 $
  \item ...
  \item $ n! = n(n - 1) $
\end{itemize}

\begin{problem}[$ (n - 2)! + 4 - 2n + 3! = ? $]
\end{problem}

\begin{proof}[\textit{ Sol. }]
  \begin{equation*}
    \begin{aligned}[t]
      n - 2 \ge 0\\
      n \ge 2
    \end{aligned}
    \qquad\qquad
    \begin{aligned}[t]
      4 - 2n \le 0\\
      4 \le 2n\\
      2 \le n
    \end{aligned}
  \end{equation*}
  \begin{align*}
    1 + 1 + 6 = 8
  \end{align*}
\end{proof}

\begin{problem}[$ 1! + 2! + 3! + ... + 999! $ birler basamağı?]
\end{problem}

\begin{proof}[\textit{ Sol. }]
  \begin{align*}
    &1 + 2 + 6 + 24 + ..0 + ..0 + ..0 = 33\\
    &\text{(10 ile bölümünden kalan)}
  \end{align*}
\end{proof}

\begin{problem}[$ 0! + 4! + 8! + 12! + ... + 100! $ onlar basamağı?]
\end{problem}

\begin{proof}[\textit{ Sol. }]
  \begin{align*}
    6! &= 720\\
    7 \times 8 &= 56\\
    8! = 8.7.6! &= 720 \times 56\\
    720 \mod 100 &= 20\\
    20 \times 56 &= 1120\\
    1120 \mod 100 &= 20\\
    1 + 24 + 20 &= 45
  \end{align*}
  \begin{align*}
    &1 + 24 + .20 + .00 + ... + .00 = 45\\
    &\text{(100 ile bölümünden kalan)}
  \end{align*}
\end{proof}

\begin{problem}[$ 23! $ içinde en çok kaç tane $ 2 $ çarpanı vardır?]
  ($ 23! = 2n \times A $ ise $ n_{max} $)
\end{problem}

\begin{proof}[\textit{ Sol. }]
  \begin{equation*}
    \begin{aligned}[t]
      23 \div 2 = 11\\
      11 \div 2 = 5\\
    \end{aligned}
    \qquad\qquad
    \begin{aligned}[t]
      5 \div 2 = 2\\
      2 \div 2 = 1
    \end{aligned}
  \end{equation*}
  \begin{align*}
    11 + 5 + 2 + 1 = 19
  \end{align*}
\end{proof}

\begin{problem}[$ 34! = 6^n \times A $, $ n_{max} = ? $]
\end{problem}

\begin{proof}[\textit{ Sol. }]
  \begin{equation*}
    \begin{aligned}[t]
      6^n = 2^n \times \underline 3^n
    \end{aligned}
    \qquad\qquad
    \begin{aligned}[t]
      34 \div 3 = 11\\
      11 \div 3 = 3\\
      3 \div 3 = 1
    \end{aligned}
  \end{equation*}
  \begin{align*}
    11 + 3 + 1 = 15
  \end{align*}
\end{proof}

\pagebreak
\begin{problem}[$ 30! = 24^n \times A $, $ n_{max} = ? $]
\end{problem}

\begin{proof}[\textit{ Sol. }]
  \begin{align*}
    24^n &= \underline 2^3n \times 3^n
  \end{align*}
  \begin{equation*}
    \begin{aligned}[t]
      30 \div 2 &= 15\\
      15 \div 2 &= 7\\
      7 \div 2 &= 3\\
      3 \div 2 &= 1
    \end{aligned}
    \qquad\qquad
    \begin{aligned}[t]
      15 + 7 + 3 + 1 &= 26\\
      3n &= 26\\
      n &\cong 8
    \end{aligned}
  \end{equation*}
\end{proof}

\begin{problem}[$ \frac{25! + 26!}{3^n} \in {Z} $, $ n_{max} = ? $]
\end{problem}

\begin{proof}[\textit{ Sol. }]
  \begin{equation*}
    \begin{aligned}[t]
      &25!(1 + 26)\\
      &25!(27)\\
      &25!(\underline 3^3) \div 3
    \end{aligned}
    \qquad\qquad
    \begin{aligned}[t]
      25 \div 3 = 8\\
      8 \div 3 = 2
    \end{aligned}
  \end{equation*}
  \begin{align*}
    8 + 2 = 10\\
    10 + 3 = 13
  \end{align*}
\end{proof}

\begin{problem}[$ 29! $ içinde kaç tane $ 10 $ çarpanı vardır?]
  ($ 29! = 10^n \times A $, $ n = ? $) ($ 29! - 1 $'in sondan kaç basamağı $ 9 $'dur?)
\end{problem}

\begin{proof}[\textit{ Sol. }]
  \begin{align*}
    10^n &= 2^n \times \underline 5^n\\
    29 \div 5 &= 5\\
    5 \div  5 &= 1
  \end{align*}
  \begin{align*}
    5 + 1 = 6
  \end{align*}
\end{proof}

\begin{problem}[$ 1 \times 1! + 2 \times 2! + 3 \times 3! + ... + 256 \times 256! $ sondan kaç basamağı $ 9 $'dur?]
\end{problem}

\begin{proof}[\textit{ Sol. }]
  \begin{align*}
    \sum_{k = 1}^{n}k \times k! = (n + 1)! - 1 \quad (257! - 1)
  \end{align*}
  \begin{align*}
    257 \div 5 &= 51\\
    51 \div 5 &= 10\\
    10 \div 5 &= 2\\
    &= 63
  \end{align*}
\end{proof}

\begin{problem}[$ \frac{24!}{2^a} $ çift sayı ise $ a_{max} = ? $]
\end{problem}

\begin{proof}[\textit{ Sol. }]
  \begin{equation*}
    \begin{aligned}[t]
      24 \div 2 = 12\\
      12 \div 2 = 6
    \end{aligned}
    \qquad\qquad
    \begin{aligned}[t]
      6 \div 2 = 3\\
      3 \div 2 = 1
    \end{aligned}
  \end{equation*}
  \begin{gather*}
    12 + 6 + 3 + 1 = 22\\
    \text{Paydanın çift olması için } a_{max} = 22 - 1 = 21
  \end{gather*}
\end{proof}

\end{document}
