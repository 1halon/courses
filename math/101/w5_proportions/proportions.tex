%%%%%%%%%%%%%%%%%%%%%%%%%%%%% Define Article %%%%%%%%%%%%%%%%%%%%%%%%%%%%%%%%%%
\documentclass{article}
%%%%%%%%%%%%%%%%%%%%%%%%%%%%%%%%%%%%%%%%%%%%%%%%%%%%%%%%%%%%%%%%%%%%%%%%%%%%%%%

%%%%%%%%%%%%%%%%%%%%%%%%%%%%% Using Packages %%%%%%%%%%%%%%%%%%%%%%%%%%%%%%%%%%
\usepackage{geometry}
\usepackage{graphicx}
\usepackage{amssymb}
\usepackage{amsmath}
\usepackage{amsthm}
\usepackage{empheq}
\usepackage{mdframed}
\usepackage{booktabs}
\usepackage{lipsum}
\usepackage{graphicx}
\usepackage{color}
\usepackage{psfrag}
\usepackage{pgfplots}
\usepackage{bm}
%%%%%%%%%%%%%%%%%%%%%%%%%%%%%%%%%%%%%%%%%%%%%%%%%%%%%%%%%%%%%%%%%%%%%%%%%%%%%%%

% Other Settings

%%%%%%%%%%%%%%%%%%%%%%%%%% Page Setting %%%%%%%%%%%%%%%%%%%%%%%%%%%%%%%%%%%%%%%
\geometry{a4paper}

%%%%%%%%%%%%%%%%%%%%%%%%%% Define some useful colors %%%%%%%%%%%%%%%%%%%%%%%%%%
\definecolor{ocre}{RGB}{243,102,25}
\definecolor{mygray}{RGB}{243,243,244}
\definecolor{deepGreen}{RGB}{26,111,0}
\definecolor{shallowGreen}{RGB}{235,255,255}
\definecolor{deepBlue}{RGB}{61,124,222}
\definecolor{shallowBlue}{RGB}{235,249,255}
%%%%%%%%%%%%%%%%%%%%%%%%%%%%%%%%%%%%%%%%%%%%%%%%%%%%%%%%%%%%%%%%%%%%%%%%%%%%%%%

%%%%%%%%%%%%%%%%%%%%%%%%%% Define an orangebox command %%%%%%%%%%%%%%%%%%%%%%%%
\newcommand\orangebox[1]{\fcolorbox{ocre}{mygray}{\hspace{1em}#1\hspace{1em}}}
%%%%%%%%%%%%%%%%%%%%%%%%%%%%%%%%%%%%%%%%%%%%%%%%%%%%%%%%%%%%%%%%%%%%%%%%%%%%%%%

%%%%%%%%%%%%%%%%%%%%%%%%%%%% English Environments %%%%%%%%%%%%%%%%%%%%%%%%%%%%%
\newtheoremstyle{mytheoremstyle}{3pt}{3pt}{\normalfont}{0cm}{\rmfamily\bfseries}{}{1em}{{\color{black}\thmname{#1}~\thmnumber{#2}}\thmnote{\,--\,#3}}
\newtheoremstyle{myproblemstyle}{3pt}{3pt}{\normalfont}{0cm}{\rmfamily\bfseries}{}{1em}{{\color{black}\thmname{#1}~\thmnumber{#2}}\thmnote{\,--\,#3}}
\theoremstyle{mytheoremstyle}
\newmdtheoremenv[linewidth=1pt,backgroundcolor=shallowGreen,linecolor=deepGreen,leftmargin=0pt,innerleftmargin=20pt,innerrightmargin=20pt,]{theorem}{Theorem}[section]
\theoremstyle{mytheoremstyle}
\newmdtheoremenv[linewidth=1pt,backgroundcolor=shallowBlue,linecolor=deepBlue,leftmargin=0pt,innerleftmargin=20pt,innerrightmargin=20pt,]{definition}{Definition}[section]
\theoremstyle{myproblemstyle}
\newmdtheoremenv[linecolor=black,leftmargin=0pt,innerleftmargin=10pt,innerrightmargin=10pt,]{problem}{Problem}[section]
%%%%%%%%%%%%%%%%%%%%%%%%%%%%%%%%%%%%%%%%%%%%%%%%%%%%%%%%%%%%%%%%%%%%%%%%%%%%%%%

%%%%%%%%%%%%%%%%%%%%%%%%%%%%%%% Plotting Settings %%%%%%%%%%%%%%%%%%%%%%%%%%%%%
\usepgfplotslibrary{colorbrewer}
\pgfplotsset{width=8cm,compat=1.9}
%%%%%%%%%%%%%%%%%%%%%%%%%%%%%%%%%%%%%%%%%%%%%%%%%%%%%%%%%%%%%%%%%%%%%%%%%%%%%%%

%%%%%%%%%%%%%%%%%%%%%%%%%%%%%%% Title & Author %%%%%%%%%%%%%%%%%%%%%%%%%%%%%%%%
\title{Types of Proportions}
\author{Halil Yiğit KOÇHAN}
\date{December 12, 2023}
%%%%%%%%%%%%%%%%%%%%%%%%%%%%%%%%%%%%%%%%%%%%%%%%%%%%%%%%%%%%%%%%%%%%%%%%%%%%%%%

\begin{document}
    \maketitle

\section{Doğru Orantı}

\begin{tikzpicture}
\begin{axis}[
  axis lines = left,
  xlabel=$ b $,
  ylabel={$ a $}
]
  \addplot [
    domain=0:1,
    samples=10,
    color=blue,
  ]
  {x};
  \addlegendentry{$ a = b \times k $, $ \frac{a}{b} = k $}
\end{axis}
\end{tikzpicture}

\begin{problem}[$ \frac{a}{0,32} = \frac{b}{0,12} = \frac{c}{0,16} $ ise $ a, b, c $ sırasıyla hangi tam sayılarla orantılıdır?]
\end{problem}

\begin{proof}[\textit{ Sol. }]
  \begin{equation*}
    \begin{aligned}[t]
      \frac{a}{\frac{32}{100}} = \frac{b}{\frac{12}{100}} = \frac{c}{\frac{16}{100}}
    \end{aligned}
    \qquad\qquad
    \begin{aligned}[t]
      \frac{100a}{32} = \frac{100b}{12} = \frac{100c}{16}\\
      \frac{a}{2} = \frac{b}{3} = \frac{c}{4}
    \end{aligned}
  \end{equation*}
  $$ a : b : c = 8 : 3 : 4 $$
\end{proof}

\pagebreak
\begin{problem}[$ 3a = 5b = 6c $ eşitliğine göre $ a, b, c $ hangi tamsayılarla orantılıdır?]
\end{problem}

\begin{proof}[\textit{ Sol. }]
  \begin{equation*}
    \begin{aligned}[t]
      &\frac{a}{\frac{1}{3}} = \frac{b}{\frac{1}{5}} = \frac{1}{\frac{1}{6}}\\
      &(10)\;(6)\quad(5)\\
    \end{aligned}
    \qquad\qquad
    \begin{aligned}[t]
      \frac{30a}{10} = \frac{30b}{6} = \frac{30c}{5}\\
      \frac{a}{10} = \frac{b}{3} = \frac{c}{4}
    \end{aligned}
  \end{equation*}
  $$ a : b : c = 10 : 6 : 5 $$
\end{proof}

\begin{problem}[$ a - 2 $ ile $ b + 1 $ D.O'dır. $ a = 3 $ iken $ b = 2 $ ise $ a = 1 $ iken $ b = ? $]
\end{problem}

\begin{proof}[\textit{ Sol. }]
  \begin{equation*}
    \begin{aligned}[t]
      \frac{a - 2}{b + 1} = k = \frac{1}{3}\\
      \frac{-1}{b + 1} = \frac{1}{3}
    \end{aligned}
    \qquad\qquad
    \begin{aligned}[t]
      -3 = b + 1\\
      b = 4
    \end{aligned}
  \end{equation*}
\end{proof}

\begin{problem}[Bir kütüğü 5 eşit parçaya ayırmak için 120 dk geçerse 13 parça için kaç saat geçer?]
\end{problem}

\begin{proof}[\textit{ Sol. }]
  $$
  \begin{cases}
    4 \text{ kesim} &120 \text{ dk}\\
    12 \text{ kesim} &x \text{ dk}
  \end{cases}
  \quad
  \begin{cases}
    x = 360 \text{dk}
  \end{cases}
  $$
\end{proof}

\begin{problem}[12 ton benzini 1 yılda 7 otomobil veya 5 otobüs harcıyor. 132 ton benzini yılda 21 otomobil ile kaç otobüs harcar?]
\end{problem}

\begin{proof}[\textit{ Sol. }]
  \begin{equation*}
    \begin{aligned}[t]
      \begin{cases}
        12t, &7 \text{ otomobil}\\
        xt, &21 \text{ otomobil}
      \end{cases}
      \quad
      \begin{cases}
        x = 36t\\
        132 - 36 = 96t
      \end{cases}
    \end{aligned}
    \qquad\qquad
    \begin{aligned}[t]
      \begin{cases}
        5 \text{ otobüs} &12t\\
        x \text{ otobüs} &90t
      \end{cases}
      \quad
      \begin{cases}
        x = 40
      \end{cases}
    \end{aligned}
  \end{equation*}
\end{proof}

\section{Ters Orantı}

\begin{tikzpicture}
\begin{axis}[
  axis lines = left,
  xlabel=$ b $,
  ylabel={$ a $}
]
  \addplot [
    domain=0:1,
    samples=10,
    color=blue,
  ]
  {1/x};
  \addlegendentry{$ x \times y = k $}
\end{axis}
\end{tikzpicture}

\begin{problem}[$ x, y, z $ sayıları sırasıyla $ 2, 3, 4 $ ile ters orantılıdır. $ x + y + z = 39 $ ise $ z = ? $]
\end{problem}

\begin{proof}[\textit{ Sol. }]
  \begin{equation*}
    \begin{aligned}[t]
      &2x = 3y = 4z = k\\
      x = &\frac{k}{2},
      y = \frac{k}{3},
      z = \frac{k}{4}\\
      &(6)\quad\;\;(4)\qquad(3)
    \end{aligned}
    \qquad\qquad
    \begin{aligned}[t]
      \frac{36k}{12} = 39\\
      k =12 \times 3 = 36
    \end{aligned}
  \end{equation*}
\end{proof}

\begin{problem}[Birbirini çeviren 3 dişli çarkın yarıçapları sırasıyla $ 8, 12, 16 $ cm'dir. Üç çarkın dönme sayısı 1 saatte toplam $ 260 $ ise küçük çark yarım saatte kaç kez döner?]
\end{problem}

\begin{proof}[\textit{ Sol. }]
  \begin{equation*}
    \begin{aligned}[t]
      \text{Yol } = \text{Tur } \times \text{Çevre}\\
      8a = 12b = 16c\\
      a + b + c = 260\\
    \end{aligned}
    \qquad\qquad
    \begin{aligned}[t]
      &\frac{k}{8} + \frac{k}{12} + \frac{k}{16} = 260\\
      &(6)\quad\!(4)\quad(3)
    \end{aligned}
  \end{equation*}
  \begin{align*}
    \frac{13k}{48} = 260\\k = 960\\
    \frac{960}{8} = 120\\
    \text{Yarım saat} = 60
  \end{align*}
\end{proof}

\section{Birleşik Orantı}

$ z, y $ ile doğru $ z $ ile ters orantılıdır.

\begin{problem}[$ x = 12 $, $ y = 3 $ iken $ z = 4 $ ise $ x = 10 $, $ y = 5 $ iken $ z = ? $]
\end{problem}

\begin{proof}[\textit{ Sol. }]
  \begin{equation*}
    \begin{aligned}[t]
      \frac{x \times z}{y} = k
    \end{aligned}
    \qquad\qquad
    \begin{aligned}[t]
      \frac{12 \times 4}{3} = k = 16
    \end{aligned}
  \end{equation*}
  $$ 16 = \frac{10 \times z}{5} = 2z, z = 8 $$
\end{proof}

$$ \frac{\text{Yapılan İş}}{\text{Diğerleri}} $$

\begin{problem}[$ 8 $ işçi $ 6 $ saat çalışarak $ 12 m^2 $ halıyı $ 18 $ günde dokuyorsa $ 6 $ işçi $ 4 $ saat çalışarak $ 36 m^2 $ halıyı dokursa kaç günde bitirir?]
\end{problem}

\begin{proof}[\textit{ Sol. }]
  \begin{align*}
    \frac{12}{8 \times 6 \times 18} &= \frac{36}{6 \times 4 \times x}\\
    x &= 18
  \end{align*}
\end{proof}

\end{document}
