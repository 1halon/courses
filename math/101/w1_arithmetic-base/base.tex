%%%%%%%%%%%%%%%%%%%%%%%%%%%%% Define Article %%%%%%%%%%%%%%%%%%%%%%%%%%%%%%%%%%
\documentclass{article}
%%%%%%%%%%%%%%%%%%%%%%%%%%%%%%%%%%%%%%%%%%%%%%%%%%%%%%%%%%%%%%%%%%%%%%%%%%%%%%%

%%%%%%%%%%%%%%%%%%%%%%%%%%%%% Using Packages %%%%%%%%%%%%%%%%%%%%%%%%%%%%%%%%%%
\usepackage{geometry}
\usepackage{graphicx}
\usepackage{amssymb}
\usepackage{amsmath}
\usepackage{amsthm}
\usepackage{empheq}
\usepackage[makeroom]{cancel}
\usepackage{mdframed}
\usepackage{booktabs}
\usepackage{lipsum}
\usepackage{graphicx}
\usepackage{color}
\usepackage{psfrag}
\usepackage{pgfplots}
\usepackage{bm}
%%%%%%%%%%%%%%%%%%%%%%%%%%%%%%%%%%%%%%%%%%%%%%%%%%%%%%%%%%%%%%%%%%%%%%%%%%%%%%%

% Other Settings

%%%%%%%%%%%%%%%%%%%%%%%%%% Page Setting %%%%%%%%%%%%%%%%%%%%%%%%%%%%%%%%%%%%%%%
\geometry{a4paper}

%%%%%%%%%%%%%%%%%%%%%%%%%% Define some useful colors %%%%%%%%%%%%%%%%%%%%%%%%%%
\definecolor{ocre}{RGB}{243,102,25}
\definecolor{mygray}{RGB}{243,243,244}
\definecolor{deepGreen}{RGB}{26,111,0}
\definecolor{shallowGreen}{RGB}{235,255,255}
\definecolor{deepBlue}{RGB}{61,124,222}
\definecolor{shallowBlue}{RGB}{235,249,255}
%%%%%%%%%%%%%%%%%%%%%%%%%%%%%%%%%%%%%%%%%%%%%%%%%%%%%%%%%%%%%%%%%%%%%%%%%%%%%%%

%%%%%%%%%%%%%%%%%%%%%%%%%% Define an orangebox command %%%%%%%%%%%%%%%%%%%%%%%%
\newcommand\orangebox[1]{\fcolorbox{ocre}{mygray}{\hspace{1em}#1\hspace{1em}}}
%%%%%%%%%%%%%%%%%%%%%%%%%%%%%%%%%%%%%%%%%%%%%%%%%%%%%%%%%%%%%%%%%%%%%%%%%%%%%%%

%%%%%%%%%%%%%%%%%%%%%%%%%%%% English Environments %%%%%%%%%%%%%%%%%%%%%%%%%%%%%
\newtheoremstyle{mytheoremstyle}{3pt}{3pt}{\normalfont}{0cm}{\rmfamily\bfseries}{}{1em}{{\color{black}\thmname{#1}~\thmnumber{#2}}\thmnote{\,--\,#3}}
\newtheoremstyle{myproblemstyle}{3pt}{3pt}{\normalfont}{0cm}{\rmfamily\bfseries}{}{1em}{{\color{black}\thmname{#1}~\thmnumber{#2}}\thmnote{\,--\,#3}}
\theoremstyle{mytheoremstyle}
\newmdtheoremenv[linewidth=1pt,backgroundcolor=shallowGreen,linecolor=deepGreen,leftmargin=0pt,innerleftmargin=20pt,innerrightmargin=20pt,]{theorem}{Theorem}[section]
\theoremstyle{mytheoremstyle}
\newmdtheoremenv[linewidth=1pt,backgroundcolor=shallowBlue,linecolor=deepBlue,leftmargin=0pt,innerleftmargin=20pt,innerrightmargin=20pt,]{definition}{Definition}[section]
\theoremstyle{myproblemstyle}
\newmdtheoremenv[linecolor=black,leftmargin=0pt,innerleftmargin=10pt,innerrightmargin=10pt,]{problem}{Problem}[section]
%%%%%%%%%%%%%%%%%%%%%%%%%%%%%%%%%%%%%%%%%%%%%%%%%%%%%%%%%%%%%%%%%%%%%%%%%%%%%%%

%%%%%%%%%%%%%%%%%%%%%%%%%%%%%%% Plotting Settings %%%%%%%%%%%%%%%%%%%%%%%%%%%%%
\usepgfplotslibrary{colorbrewer}
\pgfplotsset{width=8cm,compat=1.9}
%%%%%%%%%%%%%%%%%%%%%%%%%%%%%%%%%%%%%%%%%%%%%%%%%%%%%%%%%%%%%%%%%%%%%%%%%%%%%%%

%%%%%%%%%%%%%%%%%%%%%%%%%%%%%%% Title & Author %%%%%%%%%%%%%%%%%%%%%%%%%%%%%%%%
\title{Taban Aritmatiği}
\author{Halil Y. KOÇHAN}
\date{November 22, 2023}
%%%%%%%%%%%%%%%%%%%%%%%%%%%%%%%%%%%%%%%%%%%%%%%%%%%%%%%%%%%%%%%%%%%%%%%%%%%%%%%

\begin{document}
    \maketitle

\section{$ abcd_x $} % (fold)
\subsection{$ x > 1 $}
\subsection{}
\begin{itemize}
  \item $ a < x $
  \item $ b < x $
  \item $ c < x $
  \item $ d < x $
\end{itemize}
\subsection{$ a, b, c, d, x \in {Z} $}

\begin{problem}[$ 32_a $, $ 4a1_6 $ ise $a$ alacağı değerler?]
\end{problem}

\textit{ Sol. }
$ \cancel{1}, \cancel{2}, 3, 4, 5 $

\section{$ abc_x $}
\subsection{x çift ise}
Birler basamağının (c) durumu
\subsection{x tek ise}
$ a + b + c = d $, d'nin durumu

\section{Problems}
\begin{problem}[$ 1342_6 = ?_{10} $]
\end{problem}

\begin{proof}[\textit{ Sol. }]
  \begin{gather*}
    1342_6 = 1 \times 6^3 + 3 \times 6^2 + 4 \times 6^1 + 2 \times 6^0\\
    = 216 + 108 + 24 + 2 = 330_{10}
  \end{gather*}
\end{proof}

\begin{problem}[$ 1433_7 = ?_5 $]
\end{problem}

\begin{proof}[\textit{ Sol. }]
  \begin{gather*}
    3 \times 7^0 + 3 \times 7^1 + 4 \times 7^2 + 3 \times 7^3 = 563\\
    563 \div 5 = 112 \pmod 3\\
    112 \div 5 = 22 \pmod 2\\
    22 \div 5 = 4 \pmod 2\\
    = 4223_5
  \end{gather*}
\end{proof}

\begin{problem}[$ 243_5 + 124_5 = ? $]
\end{problem}

\begin{proof}[\textit{ Sol. }]
  \begin{gather*}
    243 + 124 = 422\\
    3 + 4 = 7 / 5 = 1 \pmod 2
  \end{gather*}
\end{proof}

\begin{problem}[4 tabanındaki en büyük 3 basamaklı sayının 5 tabanındaki eşitir kaçtır?]
\end{problem}

\begin{proof}[\textit{ Sol. }]
$ 333_4 = ?_5 $
  \begin{gather*}
    3 \times + 4^0 + 3 \times 4^1 + 3 \times 4^2 = 3 + 12 + 48 = 63\\\\
    63 \div 5 = 12 \pmod 3\\
    12 \div 5 = 2 \pmod 2\\
    = 223_5
  \end{gather*}
\end{proof}

\begin{problem}[$ 3^9 $ sayısı 3 tabanında kaç basamaklıdır?]
\end{problem}

\begin{proof}[\textit{ Sol. }]
  1 ... 000 (9 tane 0), $3_9$, 10 basamak
\end{proof}

\end{document}
