%%%%%%%%%%%%%%%%%%%%%%%%%%%%% Define Article %%%%%%%%%%%%%%%%%%%%%%%%%%%%%%%%%%
\documentclass{article}
%%%%%%%%%%%%%%%%%%%%%%%%%%%%%%%%%%%%%%%%%%%%%%%%%%%%%%%%%%%%%%%%%%%%%%%%%%%%%%%

%%%%%%%%%%%%%%%%%%%%%%%%%%%%% Using Packages %%%%%%%%%%%%%%%%%%%%%%%%%%%%%%%%%%
\usepackage{geometry}
\usepackage{graphicx}
\usepackage{amssymb}
\usepackage{amsmath}
\usepackage{amsthm}
\usepackage{empheq}
\usepackage{mdframed}
\usepackage{booktabs}
\usepackage{lipsum}
\usepackage{graphicx}
\usepackage{color}
\usepackage{psfrag}
\usepackage{pgfplots}
\usepackage{bm}
%%%%%%%%%%%%%%%%%%%%%%%%%%%%%%%%%%%%%%%%%%%%%%%%%%%%%%%%%%%%%%%%%%%%%%%%%%%%%%%

% Other Settings

%%%%%%%%%%%%%%%%%%%%%%%%%% Page Setting %%%%%%%%%%%%%%%%%%%%%%%%%%%%%%%%%%%%%%%
\geometry{a4paper}

%%%%%%%%%%%%%%%%%%%%%%%%%% Define some useful colors %%%%%%%%%%%%%%%%%%%%%%%%%%
\definecolor{ocre}{RGB}{243,102,25}
\definecolor{mygray}{RGB}{243,243,244}
\definecolor{deepGreen}{RGB}{26,111,0}
\definecolor{shallowGreen}{RGB}{235,255,255}
\definecolor{deepBlue}{RGB}{61,124,222}
\definecolor{shallowBlue}{RGB}{235,249,255}
%%%%%%%%%%%%%%%%%%%%%%%%%%%%%%%%%%%%%%%%%%%%%%%%%%%%%%%%%%%%%%%%%%%%%%%%%%%%%%%

%%%%%%%%%%%%%%%%%%%%%%%%%% Define an orangebox command %%%%%%%%%%%%%%%%%%%%%%%%
\newcommand\orangebox[1]{\fcolorbox{ocre}{mygray}{\hspace{1em}#1\hspace{1em}}}
%%%%%%%%%%%%%%%%%%%%%%%%%%%%%%%%%%%%%%%%%%%%%%%%%%%%%%%%%%%%%%%%%%%%%%%%%%%%%%%

%%%%%%%%%%%%%%%%%%%%%%%%%%%% English Environments %%%%%%%%%%%%%%%%%%%%%%%%%%%%%
\newtheoremstyle{mytheoremstyle}{3pt}{3pt}{\normalfont}{0cm}{\rmfamily\bfseries}{}{1em}{{\color{black}\thmname{#1}~\thmnumber{#2}}\thmnote{\,--\,#3}}
\newtheoremstyle{myproblemstyle}{3pt}{3pt}{\normalfont}{0cm}{\rmfamily\bfseries}{}{1em}{{\color{black}\thmname{#1}~\thmnumber{#2}}\thmnote{\,--\,#3}}
\theoremstyle{mytheoremstyle}
\newmdtheoremenv[linewidth=1pt,backgroundcolor=shallowGreen,linecolor=deepGreen,leftmargin=0pt,innerleftmargin=20pt,innerrightmargin=20pt,]{theorem}{Theorem}[section]
\theoremstyle{mytheoremstyle}
\newmdtheoremenv[linewidth=1pt,backgroundcolor=shallowBlue,linecolor=deepBlue,leftmargin=0pt,innerleftmargin=20pt,innerrightmargin=20pt,]{definition}{Definition}[section]
\theoremstyle{myproblemstyle}
\newmdtheoremenv[linecolor=black,leftmargin=0pt,innerleftmargin=10pt,innerrightmargin=10pt,]{problem}{Problem}[section]
%%%%%%%%%%%%%%%%%%%%%%%%%%%%%%%%%%%%%%%%%%%%%%%%%%%%%%%%%%%%%%%%%%%%%%%%%%%%%%%

%%%%%%%%%%%%%%%%%%%%%%%%%%%%%%% Plotting Settings %%%%%%%%%%%%%%%%%%%%%%%%%%%%%
\usepgfplotslibrary{colorbrewer}
\pgfplotsset{width=8cm,compat=1.9}
%%%%%%%%%%%%%%%%%%%%%%%%%%%%%%%%%%%%%%%%%%%%%%%%%%%%%%%%%%%%%%%%%%%%%%%%%%%%%%%

%%%%%%%%%%%%%%%%%%%%%%%%%%%%%%% Title & Author %%%%%%%%%%%%%%%%%%%%%%%%%%%%%%%%
\title{Exponential Number}
\author{Halil Yiğit KOÇHAN}
\date{December 19, 2023}
%%%%%%%%%%%%%%%%%%%%%%%%%%%%%%%%%%%%%%%%%%%%%%%%%%%%%%%%%%%%%%%%%%%%%%%%%%%%%%%

\begin{document}
    \maketitle

\begin{gather*}
  a \times x^n + b \times x^n - c \times x^n = x^n(a + b + c)\\
  x^n \times x^b = x^n + b\\
  \frac{x^n}{x^b} = x^n - b\\
  a^n \times b^n = (a \ times b)^n\\
  \frac{a}{b}^{-x} = \frac{b^x}{a^x}
\end{gather*}

\begin{problem}[$ 2x = a $ ise $ 8^x + 1 $'in $ a $ cinsinden eşiti?]
\end{problem}

\begin{proof}[\textit{ Sol. }]
  \begin{align*}
    &2^{3x} \times 2^3 = 8a^3\\
    &(a^3)
  \end{align*}
\end{proof}

\begin{problem}[$ 2^x + 2^{x+1} = 48 $ ise $ x = ? $]
\end{problem}

\begin{proof}[\textit{ Sol. }]
  \begin{gather*}
    2^x(1 + 2) = 2^{x \times 3} = 48\\
    2^x = 16\\
    x = 4
  \end{gather*}
\end{proof}

\begin{problem}[$ a = 3^x + 1 $, $ b = 3^{-x} + 1 $ ise $ \frac{a}{b} = ? $]
\end{problem}

\begin{proof}[\textit{ Sol. }]
  \begin{gather*}
    \frac{3^x + 1}{3^{-x} + 1}\\
    \frac{1}{3^x} + 1 = \frac{3^x + 1}{3^x}\\
    \frac{3^x + 1}{3x + 1} = 3^x
  \end{gather*}
\end{proof}

\begin{problem}[$ 2^{x - 1} = 5 $ ise $ 0,5^2{x+1} = ? $]
\end{problem}

\begin{proof}[\textit{ Sol. }]
  \begin{equation*}
    \begin{aligned}[t]
      2^x \times \frac{1}{2} = 5\\
      2^x = 10\\
    \end{aligned}
    \qquad\qquad
    \begin{aligned}[t]
      \frac{1}{2}^2 \times \frac{1}{2}\\
      \frac{1^x}{4^x} \times \frac{1}{2}\\
      (2^2x)
    \end{aligned}
  \end{equation*}
  $$ \frac{1}{10^2} = \frac{1}{2} = \frac{1}{100} \times \frac{1}{2} = \frac{1}{200} $$
\end{proof}

\begin{problem}[$ \frac{8^5 \times 9^4}{2^12 \times 3^6} = ? $]
\end{problem}

\begin{proof}[\textit{ Sol. }]
  $$ \frac{2^{15} \times 3^8}{2^{12} \times 3^6} = 2^3 \times 3^2 = 72 $$
\end{proof}

\begin{problem}[$ \frac{2^{2x - 1} + 4^{x + 1}}{8^{x - 1}} = ? $]
\end{problem}

\begin{proof}[\textit{ Sol. }]
  \begin{equation*}
    \begin{aligned}[t]
      &\frac{2^{2x} \times 2^{-1} + 2^{2x} \times 2^2}{2^{3x} \times 2^{-3}}\\
      & = \frac{2^{2x}(2^{-1} +  2^2)}{2^{3x} \times 2^{-3}} = \frac{9}{2}
    \end{aligned}
    \qquad\qquad
    \begin{aligned}[t]
      &= 2^{-x} \times \frac{\frac{9}{2}}{\frac{1}{8}}\\
      &= 2^{-x} \times 36
    \end{aligned}
  \end{equation*}
\end{proof}

\begin{problem}[$ 2^x = 9 $, $ 3^y = 10 $, $ 5^z = 15 $ ise $ x, y, z $ sıralaması?]
\end{problem}

\begin{proof}[\textit{ Sol. }]
  \begin{equation*}
    \begin{aligned}[t]
      3 < x < 4\\
      2 < y < 3\\
    \end{aligned}
    \qquad\qquad
    \begin{aligned}[t]
      1 < z < 2\\
      x > y > z
    \end{aligned}
  \end{equation*}
\end{proof}

\begin{problem}[a ve b sayma sayılarıdır. $ \frac{4^a \times 5^{4b}}{100} $ sayısı 19 basamaklı en küçük doğal sayıya eşit olduğuna göre $ a \times b = ? $]
\end{problem}

\begin{proof}[\textit{ Sol. }]
  \begin{gather*}
    2^{2a} \times 5^{4b} = 10^{20}\\
    a = 10\\
    b = 5\\
    a \times b = 50
  \end{gather*}
\end{proof}

\begin{problem}[$ \frac{2^x + 2^x + 2^x + 2^{x + 1}}{5^x + 5^x} = \frac{4}{25} $ ise $ x = ? $]
\end{problem}

\begin{proof}[\textit{ Sol. }]
  \begin{equation*}
    \begin{aligned}[t]
      &\frac{2^x(1 + 1 + 1 + 2)}{5^x(1 + 1)}\\
      &\frac{2^x \times 5}{5^x \times 2} = \frac{2^2}{5^2}
    \end{aligned}
    \qquad\qquad
    \begin{aligned}[t]
      \left(\frac{5}{2}\right)^3 &= \left(\frac{5}{2}\right)^x\\
      x &= 3
    \end{aligned}
  \end{equation*}
\end{proof}

\begin{problem}[$ 4^y = 32 $, $ 4^x = 8 $ ise $ \frac{x + y}{x - y} = ? $]
\end{problem}

\begin{proof}[\textit{ Sol. }]
  \begin{equation*}
    \begin{aligned}[t]
      2^{2y} = 2^5\\
      2y = 5\\
      y = \frac{5}{2}
    \end{aligned}
    \qquad\qquad
    \begin{aligned}[t]
      2^{2x} = 2^3\\
      2x = 3\\
      x = \frac{3}{2}
    \end{aligned}
  \end{equation*}
  $$ \frac{\frac{8}{2}}{\frac{-2}{2}} = -4 $$
\end{proof}

\begin{problem}[$ (x - 2)^{3x} - 2 = 1 $ ise $ x $ değerleri toplamı?]
\end{problem}

\begin{proof}[\textit{ Sol. }]
  \begin{gather*}
    x = \left\{3, \frac{2}{3}\right\}\\
    3 + \frac{2}{3} = \frac{11}{3}
  \end{gather*}
\end{proof}

\begin{problem}[$ 3^x = 5^y $ olduğuna göre $ \left(\frac{1}{3^y}\right)^{2x} + (5^y)^{\frac{1}{x}} = ? $]
\end{problem}

\begin{proof}[\textit{ Sol. }]
  \begin{equation*}
    \begin{aligned}[t]
      &3 = \frac{5^y}{x} = (5^y)^{x - 1}\\
      &5 = \frac{3^x}{y}
    \end{aligned}
    \qquad\qquad
    \begin{aligned}[t]
      &\left(\frac{1}{3^y}\right)^{2x} = \left(\frac{3^x}{y}\right)^2 = 5^2\\
      &25 + 3 = 28
    \end{aligned}
  \end{equation*}
\end{proof}

\end{document}
